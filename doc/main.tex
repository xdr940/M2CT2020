\documentclass[A4paper, punct,space,nospace, fancyhdr, fntef,UTF8]{ctexart}

\usepackage{itils}%在这里放置需要的宏包,并设置部分所需内容
\usepackage[toc,page]{appendix}
\renewcommand\appendixname{Appendix}
\renewcommand\appendixtocname{Appendix}
\renewcommand\appendixpagename{Appendix}
\begin{document}
	\pagestyle{empty}%不要页眉页脚
	%封面与诚信声明
	


\begin{figure}[htbp]
	\begin{center}
		\includegraphics[width=4.7in]{logo}%在sty文件里已经设置了图片文件夹路径了
	\end{center}
\end{figure}
\vskip 1cm
\centerline{
	\songti\zihao{2}{2020年“深圳杯”数学建模挑战赛}%反斜杠=空一格
}
\vskip 2cm
\centerline{
	\songti\zihao{2}{近地飞行器视觉稠密建图(Dense-Mapping)系统}%反斜杠=空一格
}
\vskip 1.5cm


\centerline{\songti\zihao{4}{版本编号:20200628v2}}
\vskip 1cm
\centerline{\songti\zihao{4}{2020. 08. 10}}
%表格
\begin{comment}
	\begin{flushleft}
		\centering
			\begin{tabular}{|c|l|l|c|l|l|}
				
				\hline
				\multicolumn{6}{|c|}{
					\multirow{2}{*}{
						\heiti{队伍名称:}
						\kaishu{\quad\textbf{兴奋剂}\hspace{0.5cm}}
					}
				}\\
				\multicolumn{6}{|c|}{}\\
				
				\hline
				\multicolumn{3}{|c|}{
					\multirow{2}{*}{
						\heiti{王祥通}
						%\kaishu{\quad\textbf{□口试\ □笔试\ □撰写论文 }\hspace{0.5cm}}
					}
				} & \multicolumn{3}{c|}{
					\multirow{2}{*}{
						\heiti{四川大学空天科学与工程学院}
						%\kaishu\underline{\quad\textbf{}\hspace{1cm}}
					}
				} \\
				
				\multicolumn{3}{|c|}{}& \multicolumn{3}{c|}{} 
				
				\\ \hline
				\multicolumn{3}{|c|}{
					\multirow{2}{*}{
						\heiti{None}
					%	\kaishu\underline{\quad\textbf{周广武 }\hspace{0.5cm}}
					}
				} & \multicolumn{3}{c|}{
					\multirow{2}{*}{
						\heiti{None}
						%\kaishu\underline{\quad\textbf{ }\hspace{1cm}}
					}
				} \\
				\multicolumn{3}{|c|}{}& \multicolumn{3}{c|}{}
				
				\\ \hline
			\end{tabular}
			
		\end{flushleft}
\end{comment}


\vskip 1cm
\centerline{\songti\zihao{4}{参赛组别: 创意应用}}

%\vskip 4cm
%\centerline{\heiti\zihao{2}{四\ 川\ 大\ 学\ 研\ 究\ 生\ 院\ 制 }}%反斜杠=空一格
%\vspace{10bp}
%\centerline{\zihao{3}{2019 年 \  1 月\ }}


	
	\zihao{-4}
	\tableofcontents%生成目录
	\thispagestyle{empty}%页脚不要页码
	%“目录”两个字的样式与section的样式一致,默认居中,故将设置section标题居左放置在生成目录后
	\CTEXsetup[format={\Large\bfseries}]{section}  %section标题居左
	
	
	
	\newpage
\begin{table}
	\begin{center}
		\caption{变更记录}
		\begin{tabular}{l|ccccc}
			\toprule[2pt] 
			序号   &  更改原因 & 版本&作者&更改日期&备注 \\ \hline
			00& 初始化&20200606v1&王祥通&20200606& none\\
			01&日常更新&20200628v2&王祥通&20200628&补充内容\\
			02&日常更新&20200801v3&杨帆,朱璇&20200801&补充背景以及应用部分内容\\
			03&日常更新&20200805v4&毛明洋&20200805&补充相关实验\\

			\toprule[2pt] 
		\end{tabular}
		
		\label{tab:log}
		\vspace{-0.4cm}
	\end{center}
\end{table}
	
	
	
	
	
	%%%%正文开始,页脚有页码
	\cfoot{\zihao{-5}第 \ \thepage \ 页 \ 共 \ \pageref{lastpage} 页}
	
	%lastpage为末页标签要在最后一页加上代码 \label{lastpage}%显示总页数
	
	
	
	
	%正文
	\zihao{5} 
	\pagenumbering{arabic}%页码使用阿拉伯数字
	\setcounter{page}{0}  %重新设置页码计数
	\pagestyle{fancy}

	%章节内容
	\newpage


\centerline{\fangsong\bf\zihao{-2}{无线可充电传感器网络充电路线规划}}
\addcontentsline{toc}{section}{摘要(关键词)}%加入目录


\vskip 1cm


\vskip 10bp

{
	\kaishu	
	\hspace{5bp}{\zihao{-4}\textbf{【摘要】}} 
	本文涉及的问题为旅行商问题,
	针对问题1, 我们引入自组织映射网络(Self-Organizing Map,SOM), 通过竞争合作学习的方法, 求出最优路径:
	
	$p_{0}\rightarrow p_2\rightarrow p_1\rightarrow p_9\rightarrow p_7\rightarrow p_6\rightarrow p_{14}\rightarrow p_{11}\rightarrow p_{8}\rightarrow  p_{12}\rightarrow  p_{15}\rightarrow  p_{27}\rightarrow  p_{16}\rightarrow  p_{13}\rightarrow  p_{10}\rightarrow p_{5}\rightarrow p_{3}\rightarrow p_{4}\rightarrow p_{22}\rightarrow  p_{28}\rightarrow p_{24}\rightarrow p_{23}\rightarrow p_{21}\rightarrow p_{29}\rightarrow p_{26}\rightarrow p_{25}\rightarrow p_{18}\rightarrow p_{19}\rightarrow p_{20}\rightarrow p_{17}$
	路径长度$l= 0.1143807$.
	
	针对问题2, 本文建立优化模型, 通过最小化每个结点的充电时间建模求解而得到电池电量的允许最低值的解析解.
	针对问题3, 我们将自组织映射网络方法进一步推广, 构建四个神经元环, 将问题自然的过度到多旅行商问题(Multiple Traveling Salesmen Problem, mTSP), 并求解出四条路径为:
	\begin{eqnarray}
		\nonumber
	G_A = p_0 \rightarrow p_3 \rightarrow p_{28} \rightarrow p_{24} \rightarrow p_{23} \rightarrow p_{22} \rightarrow p_{4}, \text{路径长度} l_A = 0.031713;\\ 
	\nonumber
	G_B = p_0  \rightarrow p_{17} \rightarrow p_{19} \rightarrow p_{29} \rightarrow p_{26} \rightarrow p_{25} \rightarrow p_{18} \rightarrow p_{20}, \text{路径长度}l_B = 0.035541;\\
	\nonumber
	G_C = p_{0} \rightarrow p_{1} \rightarrow  p_{9} \rightarrow  p_{7} \rightarrow  p_{6} \rightarrow  p_{14} \rightarrow  p_{11} \rightarrow   p_{7} \rightarrow  p_{2} , \text{路径长度} l_C = 0.0408157;\\
	\nonumber
	G_D = p_{0} \rightarrow  p_{10} \rightarrow  p_{12} \rightarrow  p_{15} \rightarrow  p_{27} \rightarrow  p_{16} \rightarrow  p_{13} \rightarrow  p_{5} , \text{路径长度} l_D = 0.028521; 
	\end{eqnarray}

	路程总计$l = l_A +l_B+l_C+l_D = 0.13659$, 详细见拓扑图\ref{fig:mtsp-solution}. 此外,对多路径下的电池容量问题, 我们借助问题2的模型, 给出了四条子路径下各自电池容量最低值的解析解, 其最大值则为整个系统可允许的电池最低容量.

	最后, 为了对SOM网络性能更进一步说明, 我们将SOM模型与其变种和常见的进化算法在TSP问题公开数据集上进行了对比, 结果表明在点数较少时($N \leq 100$)时, SOM类算法优势并不明显, 当点数较多时($N \geq 100)$, SOM类的方法精度与效率明显优于其他.
	 本文代码全部开源, 详情可见支撑材料或\url{https://github.com/xdr940/M2CT2020}

	
	\vskip 10bp
	
	\hspace{5bp} {\zihao{-4}\textbf{【 关键词】}} 
	旅行商问题(TSP), 多旅行商问题(mTSP), 自组织映射网络(SOM) 
}
	\section{问题重述与分析}
	\subsection{问题重述}
	有一个数据中心结点和若干传感器结点散布在地图中, 有量电动车从数据中心出发, 对传感器结点依次充电.电动车能量消耗来自路径和对结点的充电.
	\begin{itemize}
		\item 问题一: 给定数据中心结点和传感器的位置信息, 通过一辆车从数据中心结点出发, 依次经过传感器结点, 并使得路程消耗最低.
		\item 问题二:给定每个结点的位置$x_i,y_i$和能量消耗率$w$, 以及传感器电池允许的最低电量$f(mA/h)$,通过一辆车, 并沿着问题一的结果路线, 求出个传感器的电池的容量应至少是多大才能保证整个系统一直正常运行(即系统中每个传感器的电量都不会低于$f(mA/h)$.
		\item 问题三:给定每个几点的位置和能量消耗, 以及传感器允许最低点亮, 移动充电车的速度$v(m/s)$,移动充电车的充电速率$r(mA)$, 通过四辆车, 如何规划线路, 使得充电车的路上总能量消耗最小?每个传感器的电池容量应该是多大才能保证整个系统一直运行.
	\end{itemize}
	
	
	

	\subsection{模型假设}
	为了对问题分析和解答更为理性, 先对问题的一些限定做假设
	\begin{itemize}
		\item 假设充电车无限电量.
		\item 问题二中所有结点参数完全一样
		\item 可将点抽象成无向图模型
		\item 不考虑其他乱七八糟的因素.
		\item $r$为净充电速率, 不考虑掉电影响.
	\end{itemize}


	\subsection{符号限定}

	\begin{table}[h]
	\begin{center}
		\caption{队员分工}
		\begin{tabular}{l|cc}
			\toprule[2pt] 
			   &  符号 & 意义 \\ \hline
			01& $i$& 结点标号\\
			03& $(x_i,~y_i)$&结点i的坐标\\
			04& $w_i$&结点i消耗率\\
			04& $w_i$&结点i电量消耗速率\\
			04& $f$&结点电池电量底线\\
			04& $m$&结点电池电量\\

			02& $k$&充电车标号\\
			02& $v$&充电车速度\\
			02& $r$&充电车充电速率\\
			
			04& $l_{i,j}$&两个结点的距离\\
			\toprule[2pt] 
		\end{tabular}
		
		\label{tab:distribution}
		\vspace{-0.4cm}
	\end{center}
\end{table}



	\subsection{问题分析}
		三个问题, 给定的信息越来越具体, 复杂程度依次递增. 
		问题一中, 给定信息只有结点坐标以及出发点数据中心的坐标, 这构成了一个典型的旅行商问题, 通过常规解法即可.

		问题二中, 路径暂时限定为问题一的结果,以充电车出发为$t_0$时刻, 我们可求出每个结点开始充电的时刻$t_i$, 并计算每个结点从开始到充电等待的时长$T_i = t_i - t_0$, 并将维持系统一直运行考虑进去, 即形成环路, 且在数据中心处充电时间设为$T_0 = 0$. 这是一个优化问题%这个问题待解

		问题三种, 给定四辆充电车, 这里有两个目标, 一个是使得传感器电池容量最小, 另一个是让充电车上的能量损耗最小.  能量损耗
	背景和基础
		






	%chapter02
\section{符号说明与基本假设}


	%chapter03

\section{模型建立与求解}


\subsection{问题一的模型建立与求解}


\subsection{问题二的模型建立与求解}

根据问题一的求解, 已经得到环路, 现根据环路顺序, 对结点重新标号. 根据题干描述, 到整个系统稳定时, 充电车应在该环路循环跑动.

设数据中心为$p_0$点, 设充电车对$p_i$点充电时间为$T_i$, 则$p_i$点每次电量消耗时间为$T-T_i$, 消耗电量为$w_i(T-T_i)$, 则有

\begin{eqnarray}
w_i(T-T_i) \leq rT_i 
\end{eqnarray}

即
\begin{eqnarray}
\frac{w_iT}{r+w_i} \leq T_i
\end{eqnarray}
	\section{模型评价}

    本项目利用深度估计模型对航空图像序列进行视觉深度重建. 
    
    针对现有数据集对航空场景的不足, 我们提出了可拓展MinNav数据集, 相比较KITTI能更好的接近现有航空数据集比如VisDrone的数据分布. 
    针对现有模型在相对运动物体的不足, 我们在模型上提出了统计掩膜, 以此应对现有的最好模型monodepth2在航空视角上的明显缺陷.
    我们将完整的模型迁移到嵌入式平台, 并在功耗不超过30W的情况实现稳定推断, 使得以后的嵌入式应用成为可能.
     
	\section{总结讨论}


	%参考文献
	\kaishu
	\bibliographystyle{unsrt}
	\addcontentsline{toc}{section}{参考文献} %向目录中添加条目,以章/section的名义
	\bibliography{ref.bib}
	\clearpage

\clearpage
\begin{appendices}

	\section{计划和分工与团队构成}



\end{appendices}

	\label{lastpage}%%%%显示总页数
\end{document}