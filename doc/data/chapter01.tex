\section{问题重述与分析}
	\subsection{问题重述}
	有一个数据中心结点和若干传感器结点散布在地图中, 有量电动车从数据中心出发, 对传感器结点依次充电.电动车能量消耗来自路径和对结点的充电.
	\begin{itemize}
		\item 问题一: 给定数据中心结点和传感器的位置信息, 通过一辆车从数据中心结点出发, 依次经过传感器结点, 并使得路程消耗最低.
		\item 问题二:给定每个结点的位置$x_i,y_i$和能量消耗率$w$, 以及传感器电池允许的最低电量$f(mA/h)$,通过一辆车, 并沿着问题一的结果路线, 求出个传感器的电池的容量应至少是多大才能保证整个系统一直正常运行(即系统中每个传感器的电量都不会低于$f(mA/h)$.
		\item 问题三:给定每个几点的位置和能量消耗, 以及传感器允许最低点亮, 移动充电车的速度$v(m/s)$,移动充电车的充电速率$r(mA)$, 通过四辆车, 如何规划线路, 使得充电车的路上总能量消耗最小?每个传感器的电池容量应该是多大才能保证整个系统一直运行.
	\end{itemize}
	
	
	

	\subsection{模型假设}
	为了对问题分析和解答更为理性, 先对问题的一些限定做假设
	\begin{itemize}
		\item 假设充电车无限电量.
		\item 问题二中所有结点参数完全一样
		\item 可将点抽象成无向图模型
		\item 不考虑其他乱七八糟的因素.
		\item $r$为净充电速率, 不考虑掉电影响.
	\end{itemize}


	\subsection{符号限定}

	\begin{table}[h]
	\begin{center}
		\caption{队员分工}
		\begin{tabular}{l|cc}
			\toprule[2pt] 
			   &  符号 & 意义 \\ \hline
			01& $i$& 结点标号\\
			03& $(x_i,~y_i)$&结点i的坐标\\
			04& $w_i$&结点i消耗率\\
			04& $w_i$&结点i电量消耗速率\\
			04& $f$&结点电池电量底线\\
			04& $m$&结点电池电量\\

			02& $k$&充电车标号\\
			02& $v$&充电车速度\\
			02& $r$&充电车充电速率\\
			
			04& $l_{i,j}$&两个结点的距离\\
			\toprule[2pt] 
		\end{tabular}
		
		\label{tab:distribution}
		\vspace{-0.4cm}
	\end{center}
\end{table}



	\subsection{问题分析}
		三个问题, 给定的信息越来越具体, 复杂程度依次递增. 
		问题一中, 给定信息只有结点坐标以及出发点数据中心的坐标, 这构成了一个典型的旅行商问题, 通过常规解法即可.

		问题二中, 路径暂时限定为问题一的结果,以充电车出发为$t_0$时刻, 我们可求出每个结点开始充电的时刻$t_i$, 并计算每个结点从开始到充电等待的时长$T_i = t_i - t_0$, 并将维持系统一直运行考虑进去, 即形成环路, 且在数据中心处充电时间设为$T_0 = 0$. 这是一个优化问题%这个问题待解

		问题三种, 给定四辆充电车, 这里有两个目标, 一个是使得传感器电池容量最小, 另一个是让充电车上的能量损耗最小.  能量损耗
	背景和基础
		





