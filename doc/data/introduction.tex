\section{问题重述与分析}
	\subsection{问题重述}

	随着物联网的快速发展,无线传感器网络WSN(Wireless Sensor Network)在生活中的应用也越来越广泛. 无线传感器网络中包括若干传感器(Sensors)以及一个数据中心(Data Center)。传感器从环境中收集信息后每隔一段时间将收集到的信息发送到数据中心. 数据中心对数据进行分析并回传控制信息.
	影响WSN生命周期最重要的一个因素是能量. 想要让WSN能够持续不断地运转, 就必须持续为WSN提供能量. 提供能量的方式之一是能量收集(Energy Harvesting), 通过利用太阳能或风能等环境能源让传感器自行从环境中汲取能量以维持其运作, 然而这种方式提供的能量不但不稳定,而且太过于依赖环境, 一旦环境达不到条件,WSN无法从环境中汲取能量自然也就无法运转.
	
	提供能量的另外一种方式是电池供电, 并利用移动充电器定期为传感器的电池补充能量, 从而源源不断地为WSN提供稳定的能量使其正常运转. 通过这种方式供电的网络也被称为无线可充电传感器网络WRSN(wireless Rechargeable Sensor Network)

	考虑使用第二种方式供电, 为了让WRSN正常运转,移动充电器需要定期为传感器进行充电以避免其电量低于阈值. 移动充电器从数据中心出发, 以固定的速度依次经过每个传感器, 在每个传感器处停留一段时间并以固定的充电速率为传感器充电, 直到为所有传感器充电完成之后返回数据中心. 每个传感器都有\textbf{特定}的能量消耗速率, 以及\textbf{固定}的电池容量. 移动充电器的能量消耗主要有两个方面:一是为传感器节点充电所导致的正常的能量消耗; 另外一方面则是移动充电器在去为传感器充电的路上的能量消耗.

	\begin{itemize}
		\item 问题一: 给定数据中心结点和传感器的位置信息, 通过一辆车从数据中心结点出发, 依次经过传感器结点, 并使得路程消耗最低.
		\item 问题二:给定每个结点的位置$x_i,y_i$和能量消耗率$w$, 以及传感器电池允许的最低电量$f(mA/h)$,通过一辆车, 并沿着问题一的结果路线, 求出个传感器的电池的容量应至少是多大才能保证整个系统一直正常运行(即系统中每个传感器的电量都不会低于$f(mA/h)$.
		\item 问题三:给定每个几点的位置和能量消耗, 以及传感器允许最低点亮, 移动充电车的速度$v(m/s)$,移动充电车的充电速率$r(mA)$, 通过四辆车, 如何规划线路, 使得充电车的路上总能量消耗最小?每个传感器的电池容量应该是多大才能保证整个系统一直运行.
	\end{itemize}
	
	
	




	%\begin{table}[h]
	\begin{center}
		\caption{队员分工}
		\begin{tabular}{l|cc}
			\toprule[2pt] 
			   &  符号 & 意义 \\ \hline
			01& $i$& 结点标号\\
			03& $(x_i,~y_i)$&结点i的坐标\\
			04& $w_i$&结点i消耗率\\
			04& $w_i$&结点i电量消耗速率\\
			04& $f$&结点电池电量底线\\
			04& $m$&结点电池电量\\

			02& $k$&充电车标号\\
			02& $v$&充电车速度\\
			02& $r$&充电车充电速率\\
			
			04& $l_{i,j}$&两个结点的距离\\
			\toprule[2pt] 
		\end{tabular}
		
		\label{tab:distribution}
		\vspace{-0.4cm}
	\end{center}
\end{table}



	\subsection{问题分析}
		
		问题一中, 给定信息包括传感器结点坐标以及出发点数据中心的坐标, 在题干中并没有各个传感器结点对充电车的序列要求, 所以可以抽象为一个典型的旅行商(TSP)问题而非更一般的的车辆路径问题(VRP).

		问题二中, 在已知路径总距离的情况下, 充电车运行一圈时间应该是固定的, 若要使得电池容量最小, 则需要最大电量结点最小. 另外, 在系统处于临界时, 冲电车对每个结点充电时刻恰好为该结点电量耗光时刻, 即放电时间小或等于等待时间, 通过以上联系, 可以建立优化模型求解任意结点的充电最小时间.

		问题三中, 给定四辆充电车, 这里有两个目标, 一个是使得传感器电池容量最小, 另一个是让充电车上的能量损耗最小, 所以各个车辆在找到各自回路的同时应该尽量保证路过的结点数量相似. 该问题可抽象为多重旅行商问题(mTSP), 其中履行商个数对应着充电车数量即$m=4$,仓库点个数$depot= 1$.
		






		\subsection{模型假设}
		%为了对问题分析更加有方向性, 忽略本文建立模型假设如下
		\begin{itemize}
			\item 所以结点电池规格完全一样.
			\item $r$为净充电速率, 不考虑掉电影响.
			\item 传感器结点张成的为平面坐标系, 经纬度换算统一为$0.00001\deg/1m$
		\end{itemize}