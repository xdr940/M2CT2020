\section{总结讨论}

针对问题一,本文引入som-TSP模型(Sec.\ref{sec:method1}), 对充电车进行路径规划, 旨在寻找30个传感器结点组成图的哈密顿回路, 并求出结果为$L = 11.43807$ km;
针对问题二, 本文建立优化模型, 通过最小化每个结点的充电时间建模求解而得到电池电量的允许最低值的解析解;
针对问题三, 本文将SOM模型根据文献\cite{som-mtsp}进行修改, 建立som-mTSP模型, 并编程实现求解, 得到四条路径和为$l_{A+B+D+C} = 13.6592$km, 另外, 基于求得的四条最优路径, 根据问题二的优化模型, 我们通过同样方法求解得到每条路径的最低电池容量, 从而求解得到整个系统允许的最低电池容量的解析解.

由于TSP及其相关问题皆为NP-完全问题, 本文在解决TSP以及mTSP问题上, 主要借助基于人工神经网络SOM模型及其变体的"软计算"方法, 通过逼近的策略而解得最优解, 模型在一维环形SOM网络在TSP问题求解具有相当的合理性. 就解决问题方法而言, 基于进化算法的求解方法在小规模数据上表现良好, 但在大点数情况下无论其求解时间还是精度都有着比较明显的问题, 而神经网络方法的优缺点相比较前者又恰好相反. 随着结合二者优缺点的方法相继提出, 此类问题也许会渐渐得到更好的解决.

从理论和实践两个方面来看,多重旅行商问题(mTSP)都是一个重要的问题. 首先, 此类问题概括了旅行商问题(TSP), 可以进行研究以从理论角度更好地理解TSP. 另一方面, 通过合并诸如容量, 距离和时间窗口限制之类的附加侧面限制, 可以轻松地将其扩展到更具一般性的车辆路径问题(VRP). 对mTSP的深入研究无论泛化到更一般的问题还是具象为更特殊的问题, 在实际生活中都有着大量的应用背景, 研究价值巨大.

%本文涉及代码与模型, 皆为队员原创


