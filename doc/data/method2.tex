%chapter03

\newpage
\section{问题二的模型建立与求解}
\label{sec:method2}
\subsection{问题二符号说明}

\begin{table}[h]
	\begin{center}
		\caption{问题二符号说明}
		\begin{tabular}{l|c}
			\toprule[2pt] 
			    符号 & 意义 \\ \hline
			 $i$& 传感器结点标号\\
             $T_i$&结点i充电时间\\
             $\bar{T}_i$&结点i放电时间\\
             $T$&充电系统运行一个循环的时间\\
			 $w_i$&结点i电量消耗功率\\
			 $f$&结点电池电量底线\\
			 $c_i$&结点电池电量\\
			 $v$&充电车行进速度\\
			 $r$&充电车充电功率\\
			
			\toprule[2pt] 
		\end{tabular}
		
		\label{tab:distribution}
		\vspace{-0.4cm}
	\end{center}
\end{table}


\subsection{问题二的模型建立与求解}
根据问题一的求解, 已经得到环路, 根据问题二题干描述, 我们可以假设到整个系统稳定时, 充电车应在该环路循环跑动, 且运行一圈时间固定不变. 在系统运行时, 电池可能的最小容量即为耗电量最大的结点, 而对于充电速率统一为$r$, 耗电量最大的结点充电时间最长的结点. 又因为所有节点的电池容量统一, 故电池容量最小应为:

\begin{eqnarray}
    c_{min} = \mathop{Max} T_i \cdot r + f
    \label{eq:cmin}
\end{eqnarray}

其中$c_{min}$为电池最小容量, $T_i$为结点$p_i$的充电时间, $r$为充电功率, $f$为电池最低电量.

上式中, 充电时间$T_i$为变量, 故可建立充电时间模型, 求出每个结点最小充电时间, 则问题解决.

设数据中心为$p_0$点, 电动车跑完一轮时间为$T$, 每个结点充的电量为$c_i$, 充电车对$p_i$点充电时间为$T_i$, 则$p_i$点每个循环电量消耗时间为$\bar{T}_i = T-T_i$, 放电功率为$w_i$,故消耗电量为$w_i(T-T_i)$, 则可列方程:

\begin{eqnarray}
c_i = w_i \bar{T_i} = w_i(T-T_i)\leq rT_i 
\end{eqnarray}

将不等式移项, 将$T_i$分离出, 则有

\begin{eqnarray}
\nonumber
\quad &\frac{w_iT}{r+w_i} \leq T_i&\\
\nonumber
\quad &\frac{w_i}{r+w_i} T_{r} + \frac{w_i}{r+w_i} \sum\limits_j T_j - T_i \leq 0&
\end{eqnarray}

令$\frac{w_i}{r+w_i} = a_i, \textbf{a} = [a_1, a_2, \dots a_n], ~\textbf{T} = [T_1,\dots T_i \dots T_n]$
则上式可化为
\begin{equation}       %开始数学环境
    \left[                 %左括号
      \begin{array}{cccc}   %该矩阵一共3列,每一列都居中放置
        a_1-1 & a_2 & \dots& a_n \\  %第一行元素
        a_1 & a_2 -1 & \dots& a_n \\  %第二行元素
        &&\dots&\\
        a_1&a_2&\dots&a_n-1\\
      \end{array}
    \right]                 %右括号
    \left[
        \begin{array}{cccc}
         T_1\\
         T_2\\
         ... \\
         T_{n}
        \end{array}
    \right ]
    +
    \left[
    \begin{array}{cccc}
     a_{1}\\
     a_{2}\\
     ... \\
     a_{n}
    \end{array}
    \right ]
    T_r \leq 0
    \end{equation}

  

    \begin{equation}     
        \left\{
            \left[                 
                \begin{array}{cccc}   
                1 & 0 & \dots& 0 \\  
                0 & 1  & \dots& 0 \\ 
                &&\dots&\\
                0&0&\dots&1\\
                \end{array}
            \right] 
            -
            \left[                 
            \begin{array}{cccc}   
                a_1 & a_2 & \dots& a_n \\ 
                a_1 & a_2  & \dots& a_n \\  
                &&\dots&\\
                a_1&a_2&\dots&a_n\\
            \end{array}
            \right]    
        \right\}   
        \left[
            \begin{array}{cccc}
             T_1\\
             T_2\\
             ... \\
             T_{n}
            \end{array}
        \right ]
        \geq
        \left[
            \begin{array}{cccc}
            a_{1}\\
            a_{2}\\
            ... \\
            a_{n}
            \end{array}
        \right ]
        T_r 
        \end{equation}

即
\begin{eqnarray}
  (\textbf{I}- \textbf{1}^T\textbf{a} )\textbf{T} \geq \textbf{a} T_r
\end{eqnarray}
所以
\begin{eqnarray}
  \textbf{T} \geq   (\textbf{I}- \textbf{1}^T\textbf{a} )^{-1}\textbf{a} T_r
\end{eqnarray}
因为$T_i \geq 0$所以有:
\begin{eqnarray}
    \mathop{Max}T_i = \| (\textbf{I}- \textbf{1}^T\textbf{a} )^{-1}\textbf{a} T_r\|_{\infty}
\end{eqnarray}
这里$\|\cdot\|_\infty$为无穷范数, \textbf{1}为n维全1向量. 再将$ T_r = l/v$ 带入,联合方程\ref{eq:cmin} 得到

\begin{eqnarray}
    c_{min} =r \| (\textbf{I}- \textbf{1}^T\textbf{a} )^{-1}\textbf{a} \frac{l}{v}\|_{\infty} +f
\end{eqnarray}

综上, 求得整个充电系统中电池最小容量为$ c_{min} =r \| (\textbf{I}- \textbf{1}^T\textbf{a} )^{-1}\textbf{a} \frac{l}{v}\|_{\infty} +f $


%其中, $T = T_{charge} + T_{run}$, $T_{charge} = \sum \limits_{i} T_i  $. $T_{run} = l/v $






